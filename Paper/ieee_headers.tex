%%%%%%%%%%%%%%%%%%%%%%%%%%%%%%%%%%%%%%%%%%%%%%%%%%%%%%%%%%%%%%%%%%%%%%%%%%%%%%%%
%2345678901234567890123456789012345678901234567890123456789012345678901234567890
%        1         2         3         4         5         6         7         8

\documentclass[letterpaper, 10 pt, conference]{ieeeconf}  % Comment this line out if you need a4paper


\IEEEoverridecommandlockouts                              % This command is only needed if 
                                                          % you want to use the \thanks command
\overrideIEEEmargins                                      % Needed to meet printer requirements.

% See the \addtolength command later in the file to balance the column lengths
% on the last page of the document

% The following packages can be found on http:\\www.ctan.org
%\usepackage{graphics} % for pdf, bitmapped graphics files
%\usepackage{epsfig} % for postscript graphics files
%\usepackage{mathptmx} % assumes new font selection scheme installed
%\usepackage{times} % assumes new font selection scheme installed
\usepackage{amsmath} % assumes amsmath package installed
\usepackage{amssymb}  % assumes amsmath package installed


\usepackage{verbatim}



\usepackage{cite}

 \usepackage{mathtools}
\usepackage[ruled,linesnumbered]{algorithm2e}
\newcommand{\norm}[1]{\left\lVert #1 \right\rVert}
\usepackage{amssymb}

% \usepackage{subcaption}

% \usepackage{tikz}
% \usetikzlibrary{shapes,arrows}


\usepackage{graphicx}
% \usepackage{caption}


\usepackage{comment}


% \usepackage{algorithm}% http://ctan.org/pkg/algorithms
% \usepackage{algpseudocode}% http://ctan.org/pkg/algorithmicx
% \floatname{algorithm}{Procedure}
% \renewcommand{\algorithmicrequire}{\textbf{Input:}}
% \renewcommand{\algorithmicensure}{\textbf{Output:}}

% \usepackage{ulem}
\usepackage{fancybox,color}
\def\ani{\textcolor{red}}
\def\din{\textcolor{blue}}
\def\changedsubh{\textcolor{blue}}

\definecolor{grey}{rgb}{0.5,0.5,0.5}


\usepackage[utf8]{inputenc}
\usepackage[english]{babel}
 
\newtheorem{theorem}{Theorem}
\newtheorem{proposition}{Proposition}
\usepackage{amsmath} 
